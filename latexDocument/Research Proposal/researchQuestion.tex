\section{Research question}
The goal of this project is to create a neural network that can control a robotic arm so that it is able to grasp an object that it has seen by using a 3D camera. 
The main algorithm for learning to control the arm will be the CACLA algorithm, we will use a CNN for the perception of objects that the arm needs to grasp. 
The project will be divided into two aspects, control and perception. For the control part the focus will be on creating a neural network that can grasp an object given by an operator manually. The perception part 
will focus on the detection of an object and extracting the location and orientation needed to grasp the object. The final stage is to make the control part depending on the perception part so it can autonomously grasp objects
that are seen by the camera. Another aspect in this research project is to look at different architectures. Instead of having one neural network that controls the arm, perhaps a better result or faster training can be 
achieved by having multiple networks where each network controls a single joint or smaller groups of joints, a network used for approaching the object and one network used for the actual grasping part, separate the CNN and 
the control neural network or combine both CNN and control neural network into one big neural network. \\
We will compare this method to an existing inverse kinematic controller that is being used by the MoveIt package. We will look at the speed, the time it takes the inverse kinematics to calculate and execute a grasp versus the immediate
execution of the neural network controller. We will look at accuracy, is the controller able to grasp the object and how good are the grasps of each controller, whether it is holding the object correctly or just barely grasped it. 
Also we will look at the robustness, how well do the controllers perform in different situations. \\
We will first focus on just grasping a single object from a scene so we don't require an advanced object recognizer to separate objects. If these methods are working correctly and there is time left we can focus on muliple objects in 
a scene and an object recognizer to determine locations for each object.\\ \\
\textbf{The main research question}:    
\begin{itemize}
  \item  How does deep reinforcement learning compare to a much simpler (inverse) kinematics controller for grasping objects? Is there a gain in speed, accuracy or robustness?
\end{itemize}                                
\textbf{The sub-questions}: 
\begin{itemize}
  \item Which architecture results in the best performance? 
  \item Does training in simulation result in correct behaviour on the real robot?
\end{itemize}
