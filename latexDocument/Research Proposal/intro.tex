\section{Introduction}
Service robots are slowly becoming more popular, mostly small robots with limited functionality like vacuum cleaner robots. Bigger service robots that could help in a household environment are still much in development. 
These service robots need to be able to perform many complex tasks, in all kinds of environments, while doing them safely by not colliding with objects, humans and itself.   
Tasks like navigating, speech recognition, following/recognizing humans, object detection/recognition and manipulation are all important parts for a service robot, 
and these tasks are tested in competitions like the RoboCup and RoCKin [REFS]. The difficulty in programming these robots is that they need to operate in all kinds of different environments, no household is same, and this makes it hard
to make sure that the robot can operate safely and correctly in situations that is has never seen before. This means that the robots needs to have intelligent behaviour so it can cope with any environment it encounters. \\
In this project we will focus on the manipulation and perception part of a service robot by making use of reinforcement learning so that the robot has learned how to grasp an object that it has seen with its camera.  
Common used approaches for grasping objects with service robots are currenlty by using simple cartesian control, or by making use of a planner that creates a path for the arm to traverse so it can grasp an object. 
The downside of using a planner is that it requires to make new calculations every time it needs to grasp an object, with the downside that sometimes it can not find a valid plan, or it takes to long to find a valid plan
for the arm to move by. \\
**Something about deeplearning** \\
**Something about continues control vs discrete control** \\

% about fast grasping planning
% Stuckler, J., Ste ens, R., Holz, D., Behnke, S.:  Ecient 3D object perception and
%grasp planning for mobile manipulation in domestic environments.  Robotics and
%Autonomous Systems
%61
%(2013) 1106{1115


%For the perception part we only focus on object detection and localization, for the manupulation we assume to only pick up objects that are 
%are easy to pick up, objects that have a basic geometric shape like a box or cylinder (e.g. a coca-cola can or a small bottle). 
%
% - Two methods, simple cartesian control, complex angular control with planning. \\
% - Arm should learn how to grasp objects.  \\
% - Important that not only work in simulation but also on real arm, real world. \\
 
% - Reinforcement learning, continues action space, Cacla, google's DDPG. \\
% - 



%-Research question: Is it possible for a robotic arm to grasp an object by using reinforcement learning?
%  - Which architecture gives the best results for grasping an object?
